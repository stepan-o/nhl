%! Author = Stepan Oskin
%! Date = 2019-07-19

% Preamble
\documentclass[11pt]{article}

% Packages
\usepackage{amsmath}

% Document
\begin{document}


    \title{OSEMN methodology\\
    Excerpts from How To Work Through A Problem Like A Data Scientist \\
    By Jason Brownlee\cite{Brownlee2014} \\
    and other sources}


    \author{Stepan Oskin}

    \maketitle

    \begin{abstract}

    In a 2010 post Hilary Mason and Chris Wiggins described the OSEMN process as a taxonomy of tasks that a data scientist should feel comfortable working on.

    The title of the post was \textit{A Taxonomy of Data Science} on the now defunct dataists blog\cite{Mason2010}.
    This process has also been used as the structure of a recent book, specifically \textit{Data Science at the Command Line: Facing the Future with Time-Tested Tools} by Jeroen Janssens published by O\’Reilly\cite{Janssens}.

    In this document, we take a closer look at the OSEMN process for working through a data problem.
    \end{abstract}



    \bibliography{osemn}
    \bibliographystyle{ieeetr}

\end{document}
