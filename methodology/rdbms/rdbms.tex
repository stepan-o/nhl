%! Author = Stepan Oskin
%! Date = 2019-07-22

% Preamble
\documentclass[11pt]{article}

% Packages

% Document
\begin{document}

    \title{Relational databases \\
    Description of methodology \\
    Based on a course offered by DataCamp \\
    \textit{Introduction to Relational Databases} \\
    and other sources}

    \author{Stepan Oskin}

    \maketitle

    \begin{abstract}
        A relational database models real life entities, such as NHL players and NHL teams, by storing them in tables.
        Each table must contain data from a single entity type.
        This reduces redundancy by storing entities only once.
        A database can then be used to model relationships between entities and to preserve data quality through such concepts as constraints, keys, and referential integrity.
        SQL, or Structured Query Language, can be used for querying, as well as building and maintaining databases.
    \end{abstract}

    \section{Relational Database} \label{sec:rdb}

    A relational database models real life entities, such as NHL players and NHL teams, by storing them in tables.
    Each table must contain data from a single entity type (\textit{e.g.}, NHL Players, NHL teams, \textit{etc.})
    This reduces redundancy by storing entities only once \textemdash for example, there only needs to be one row of data containing details of a certain NHL team.
    Lastly, a database can be used to model relationships between entities.
    For instance, an NHL player could have played for multiple NHL teams, while an NHL team includes many NHL players on their roster.

    \vspace{5mm}

    Characteristics of a relational database:
    \begin{itemize}
        \item real-life \textit{entities} become \textit{tables}
        \item reduced redundancy
        \item data integrity by \textit{relationships}
    \end{itemize}

    Relational databases can help preserve data quality through such concepts as:

    \begin{itemize}
        \item constraints
        \item keys
        \item referential integrity
    \end{itemize}

    SQL, or \textbf{Structured Query Language}, can be used for querying, as well as building and maintaining databases.

    \section{\texttt{information\_schema} database} \label{sec:info_schema}

    \texttt{information\_schema} database is available by default in PostgreSQL and presents a \textit{meta database} that holds information about a relational database in PostgreSQL .
    \texttt{information\_schema} is not PostgreSQL specific, and is also available in other database management systems, such as MySQL or SQL Server.
    The \texttt{information\_schema} database holds various information in different tables, for example, in a \texttt{tables} or \texttt{columns} tables.

\end{document}